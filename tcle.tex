\documentclass[12pt,a4paper]{article}
\usepackage[utf8]{inputenc}
\usepackage[portuguese]{babel}
\usepackage[T1]{fontenc}
\usepackage{amsmath}
\usepackage{amsfonts}
\usepackage{amssymb}
\usepackage{graphicx}
\usepackage{lmodern}
\usepackage{textcomp}
\usepackage{xcolor}
\usepackage[left=2cm,right=2cm,top=2cm,bottom=2cm]{geometry}
\author{Gilmar Gomes do Nascimento}
\title{Termo de Consentimento Livre e Esclarecido (TCLE)}
\definecolor{labelcolor}{RGB}{70, 70, 70}
\definecolor{linecolor}{RGB}{150, 150, 150}

% Custom command for form fields
\newcommand{\formfield}[2][1]{%
	\textcolor{labelcolor}{\textbf{#2}} \\[0.3cm]
	\textcolor{linecolor}{\makebox[\linewidth]{\hrulefill}} \\[#1cm]
}

% Custom command for two-column fields
\newcommand{\twocolfield}[4]{%
	\begin{minipage}[t]{#1\textwidth}
		\formfield[#2]{#3}
	\end{minipage}%
	\hfill
	\begin{minipage}[t]{#4\textwidth}
		\formfield[#2]{#3}
	\end{minipage}
}

% Custom command for three-column fields
\newcommand{\threecolfield}[6]{%
	\begin{minipage}[t]{#1\textwidth}
		\formfield[0.8]{#2}
	\end{minipage}%
	\hfill
	\begin{minipage}[t]{#3\textwidth}
		\formfield[0.8]{#4}
	\end{minipage}%
	\hfill
	\begin{minipage}[t]{#5\textwidth}
		\formfield[0.8]{#6}
	\end{minipage}
}
\begin{document}

\begin{center}
\textbf{TERMO DE CONSENTIMENTO LIVRE E ESCLARECIDO(TCLE)}
\end{center}

\vspace{2\baselineskip} % Multiples of line height (more robust)
\noindent
\textbf{Título da Pesquisa:} Uso de logs no auxílio no ensino de programação \\
\textbf{Instituição Proponente:} Universidade Tecnológica Federal do Paraná(UTFPR-CT)\\
\textbf{Programa:} Programa de Pós-Graduação em Computação Aplicada (PPGCA/UTFPR-CT)\\
\textbf{Endereço:} Av. Sete de Setembro, 3165 - Rebouças, Curitiba - PR, CEP 80230-901 \\
\textbf{Telefone:} (41) 3310-4524 \\

\vspace{\baselineskip}
\noindent \textbf{Pesquisador Responsável:} Laudelino Cordeiro Bastos\\
\textbf{Vínculo Institucional:} Docente no PPGCA/UTFPR-CT \\
\textbf{Endereço:} Av. Sete de Setembro, 3165 - Rebouças, Curitiba - PR, CEP 80230-901 \\
\textbf{E-mail:} bastos@utfpr.edu.br \\
\textbf{Telefone:} 41 9972-8239 \\

\vspace{\baselineskip}
\noindent\textbf{Pesquisador:} Gilmar Gomes do Nascimento \\
\textbf{Vínculo Institucional:} Professor EBTT IFAM/CBDA - Discente PPGCA/UTFPR-CT \\
\textbf{Endereço:} Rua Fontenele de Castro, sn \\
\textbf{E-mail:} gilmar.nascimento@ifam.edu.br \\
\textbf{Telefone:} (97) 999149-5919 \\

\vspace{\baselineskip}
\noindent
\textbf{Pesquisadora:} Maria Claudia Emer \\
\textbf{Vínculo Institucional:} Docente no PPGCA/UTFPR-CT\\
\textbf{Endereço:} Av. Sete de Setembro, 3165 - Rebouças, Curitiba - PR, CEP 80230-901 \\
\textbf{E-mail:} mciemer@gmail.com \\
\textbf{Telefone:} 41 9282-4938
\vspace{\baselineskip}

\section{AO PARTICIPANTE}
\subsection{Apresentação da pesquisa}

\noindent Você está sendo convidado(a) a participar de uma pesquisa científica intitula ``O uso de logs no auxílio de ensino de programação". Esta pesquisa é de responsabilidade de Laudelino Cordeiro Bastos, Gilmar Gomes do Nascimento e Maria Claudia Emer, pesquisadores(as) vinculados(as) ao Programa de Pós-Graduação em Computação Aplicada (PPGCA) da Universidade Tecnológica Federal do Paraná, campus Curitiba(UTFPR-CT).

Este documento é nomeado Termo de Consentimento Livre e Esclarecido(TCLE). Nele estão contidas as principais informações sobre a pesquisa. Para que possa assegurar seus direitos quanto à participação no estudo é importante e necessário que você mantenha sob a sua guarda uma cópia deste termo, sendo o mesmo aplicável aos demais questionários da pesquisa. Assim, lhe orientamos para que, sempre após a leitura e preenchimento dos termos e questionários que lhe forem apresentados, você realize o download e impressão de documento/página do navegador. 

Contexto e justificativa da pesquisa: O uso de dispositivos deixam rastros, esses logs geralmente são de acesso ou de uso de algum software, este trabalho busca armazenar logs no ensino-aprendizagem de desenvolvimento de códigos nas aulas de programação no laboratório de informática utilizando um editor de texto e ao plugin que será adicionado com informações básicas de acesso, como nome de usuário, uma senha e o código da disciplina. 

\section{Objetivos da pesquisa}

O objetivo geral deste trabalho é desenvolver e implementar uma extensão para coleta e análise de logs durante o processo de ensino-aprendizagem de programação, com o intuito de fornecer métricas que auxiliem na identificação de dificuldades dos alunos, na personalização do ensino e no aprimoramento das práticas pedagógicas

\section{Participação na pesquisa}

Sua participação nesta pesquisa é voluntária (sem remuneração e ou recompensas) e não obrigatória. Você possui plena liberdade para decidir se deseja participar dela, não sofrendo qualquer tipo de penalização caso opte por não participar. 

O método utilizado será a Pesquisa de Ciência do Design. Assim os(as) estudantes que aceitarem participar serão divididos em grupos. Os grupos preencherão questionários e participarão de um momento de ensino e aprendizado do conteúdo os dados separados em arquivos de textos com os logs e na próxima etapa os dados estarão aglomerados sem identificação em uma base de dados maior com outros dados. 

O experimento será realizado em laboratórios de informática do IFAM-CBDA durante o horário das disciplinas ``Algoritmo e Lógica de Programação" e ``Estrutura de Dados".


\section*{ESCLARECIMENTOS SOBRE O COMITÊ DE ÉTICA EM PESQUISA}
 O Comitê de Ética em Pesquisa envolvendo Seres Humanos (CEP) é constituído por uma equipe de profissionais com formação multidisciplinar que está trabalhando para assegurar o respeito aos seus direitos como participante da pesquisa. Ele tem por objetivo avaliar se a pesquisa não está sendo realizada da forma como você foi informado ou que você está sendo prejudicado de alguma forma, entre em contato com o Comitê de Ética em Pesquisa envolvendo Seres Humanos do Instituto Federal de Educação, Ciência e Tecnologia do Amazonas. \textbf{E-mail:} cepsh.ppgi@ifam.edu.br
\newpage
\subsection*{CONSENTIMENTO}

Declaro que, após a leitura deste termo, reflexão e um tempo razoável para a tomada de decisão, aceito livre e voluntariamente participar desta pesquisa.\\[1cm]

\noindent
\begin{minipage}{\textwidth}
	\formfield[1.2]{Nome completo:}
\end{minipage}

% RG and Birth date (two columns)
\noindent
\begin{minipage}[t]{0.48\textwidth}
	\formfield[1]{RG:}
\end{minipage}%
\hfill
\begin{minipage}[t]{0.48\textwidth}
	\formfield[1]{Data de Nascimento:}
\end{minipage}

% Phone number
\noindent
\begin{minipage}{\textwidth}
	\formfield[1.2]{Celular:}
\end{minipage}

% Address (with two lines)
\noindent
\begin{minipage}{\textwidth}
	\textcolor{labelcolor}{\textbf{Endereço:}} \\[0.3cm]
	\textcolor{linecolor}{\makebox[\textwidth]{\hrulefill}} \\[0.3cm]
	\textcolor{linecolor}{\makebox[\textwidth]{\hrulefill}} \\[1.2cm]
\end{minipage}

% ZIP, City, State (three columns)
\noindent
\begin{minipage}[t]{0.25\textwidth}
	\formfield[0.8]{CEP:}
\end{minipage}%
\hfill
\begin{minipage}[t]{0.45\textwidth}
	\formfield[0.8]{Cidade:}
\end{minipage}%
\hfill
\begin{minipage}[t]{0.25\textwidth}
	\formfield[0.8]{Estado:}
\end{minipage}

\vspace{1cm}

% Signature field
\noindent
\begin{minipage}{\textwidth}
	\formfield[0.5]{Assinatura:}
\end{minipage}

\vspace{0.5cm}

% Date field for signature
\noindent
\begin{minipage}[t]{0.6\textwidth}
	\formfield[0.5]{Local:}
\end{minipage}%
\hfill
\begin{minipage}[t]{0.35\textwidth}
	\formfield[0.5]{Data:}
\end{minipage}
\end{document}