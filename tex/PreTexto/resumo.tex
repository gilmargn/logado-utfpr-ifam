%%%% RESUMO
%%
%% Apresentação concisa dos pontos relevantes de um texto, fornecendo uma visão rápida e clara do conteúdo e das conclusões do
%% trabalho.

\begin{resumoutfpr}%% Ambiente resumoutfpr
No desenvolvimento de software, é comum utilizar métricas para avaliar a qualidade do código e o tempo 
despendido em sua produção. 
No ambiente profissional, essa análise ocorre principalmente por meio de \textit{pull requests}, enquanto no 
contexto acadêmico, a avaliação costuma ser feita por meio de envios pontuais (como e-mails ou sistemas de 
gestão de aprendizagem - LMS). Embora os LMS registrem \textit{logs }de atividades e eventos, há uma lacuna na 
coleta estruturada de dados sobre o processo de aprendizagem em programação, especialmente em ambientes como 
laboratórios de informática.
Este trabalho propõe a criação de um \textit{dataset} com registros (logs) do processo de desenvolvimento de 
código, capturando o rastro da aprendizagem em tempo real. Diferentemente de abordagens convencionais, que 
dependem de Sistemas de Gestão de Aprendizagem, a coleta será realizada por meio de um \textit{plugin} 
integrado a um editor de código amplamente utilizado, voltado tanto para programação quanto para documentação. 
Espera-se fornecer uma base de dados sobre o ensino-aprendizagem de programação que possa auxiliar na 
identificação de dificuldades dos estudantes, na personalização do ensino e no aprimoramento das práticas 
pedagógicas.
\end{resumoutfpr}

%De acordo com a NBR 6028:2021, a apresentação gráfica deve seguir o padrão do documento no qual o resumo está inserido. Para definição das palavras-chave (e suas correspondentes em inglês no abstract) consultar em Termo tópico do Catálogo de Autoridades da Biblioteca Nacional, disponível em: http://acervo.bn.gov.br/sophia_web/autoridade