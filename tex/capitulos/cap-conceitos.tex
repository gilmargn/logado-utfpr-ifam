\chapter{Fundamentação Teórica}
\label{cap:fundamentacao}

Este capítulo apresenta a fundamentação teórica necessária para o entendimento desta dissertação. Inicialmente, contextualiza os desafios do ensino de programação e introduz o conceito de \textit{logging} como uma prática crucial da Engenharia de Software. Em seguida, explora o uso de \textit{logs} no contexto educacional, detalha a taxonomia e os tipos de informações registradas, e apresenta a metodologia proposta para a análise desses dados. Por fim, é descrita a extensão \textit{Logado}, que constitui a contribuição prática deste trabalho.

\section{Desafios no Ensino de Programação e a Lacuna Instrucional sobre Logging}
\label{sec:desafios-lacuna}

O ensino de programação transcende a mera transmissão de conhecimentos técnicos, abrangendo aspectos como boas práticas de desenvolvimento, integridade acadêmica e a compreensão do ciclo de vida do software. No entanto, uma análise da literatura pedagógica fundamental revela uma lacuna significativa.

Para mapear o estado da arte, consultou-se os conteúdos de livros-texto amplamente adotados globalmente (como os de Sommerville e Pressman) e a principal referência nacional, ``Engenharia de Software Moderna'' de Marco Tulio Valente. Constatou-se que, embora essas obras abordem tópicos como testes e qualidade de software, não dedicam seções ou discussões substantivas ao \textit{logging} como uma prática de engenharia a ser ensinada, com suas próprias estratégias e melhores práticas \cite{gu2022logging}.

Esta omissão é notável dada a criticalidade dos \textit{logs} na indústria e contrasta com o potencial educacional da análise dos rastros digitais que os alunos naturalmente geram durante a codificação. No Brasil, o tema é ainda mais incipiente; uma busca no portal da SBC com os termos \texttt{log OR logging AND "ensino de programação"} retorna poucos ou nenhum resultado, indicando um nicho de pesquisa promissor.

\section{Logging: Conceitos e Aplicações na Engenharia de Software}
\label{sec:conceitos-logging}

Para estabelecer uma base terminológica unificada, adota-se a taxonomia de \cite{gu2022logging}, que define os conceitos fundamentais:
\begingroup
\renewcommand{\arraystretch}{1.2}
\begin{tabular}{p{0.3\textwidth}p{0.3\textwidth}p{0.3\textwidth}}
    • Log statement       & • Log message     & • Log \\
    • Log level           & • Log content     & • Log location \\
    • Log placement       & • Logging         & • Logging practice \\
\end{tabular}
\endgroup
\vspace{2em}

Segundo \cite{rice1983use}, o monitoramento de sistemas pode ser entendido como um registro automático de transações. A importância dos \textit{logs} é vasta, sendo utilizados em tarefas como detecção de anomalias, depuração, diagnóstico de desempenho e modelagem de comportamento de sistemas \cite{fu_where_2014}. Estudos como o de \cite{yang2021interview} confirmam que os desenvolvedores os utilizam primariamente para análise de problemas, buscando informações como a propagação de erros, \textit{timestamps} e o \textit{dataflow} entre componentes.

\section{A Análise de Logs como Ferramenta Educacional}
\label{sec:logs-educacionais}

Aplicar a análise de \textit{logs} no ensino de programação significa transpor essa prática da manutenção de sistemas para a análise do processo de aprendizagem. Os algoritmos de ensino não produzem apenas resultados; geram \textit{logs} e rastros digitais detalhados que, quando analisados, podem fornecer insights sobre a curva de aprendizagem.

Como sugerido por \cite{silva2022previsao}, variáveis preditoras podem ser extraídas do código, tais como: número de linhas lógicas, uso de estruturas de repetição, quantidade de identificadores, funções utilizadas, entre outras. Esta abordagem está alinhada com a \textit{Learning Analytics}, cuja meta é melhorar a qualidade do ensino e da aprendizagem através da análise de dados sobre o comportamento dos estudantes \cite{huang2020predicting}. Os \textit{logs} de programação funcionam, assim, como um registro de aprendizagem (\textit{learning record}) digital, revelando a experiência pessoal e as reflexões do estudante \cite{du2005learning}.

\section{Metodologia para Análise de Logs Educacionais}
\label{sec:metodologia-analise}

A análise dos dados coletados será realizada em três etapas principais:

\subsection{Coleta e Pré-processamento}
Os dados brutos são gerados a partir da interação dos estudantes com o ambiente de programação, capturando eventos como edições de código, compilações e erros. Esta fase envolve:
\begin{itemize}
    \item \textbf{Registro de Logs}: Captura de \textit{timestamp}, trechos de código editados e eventos de desenvolvimento.
    \item \textbf{Filtragem e Anonimização}: Substituição de dados sensíveis por identificadores únicos para garantir a privacidade.
\end{itemize}

\subsection{Geração de Indicadores Educacionais}
Os \textit{logs} brutos são transformados em métricas educacionais significativas (e.g., tempo para resolver um erro, padrões de tentativa-e-erro), criando um \textit{dataset} para pesquisa. Este \textit{dataset} permitirá estudos sobre a eficácia de metodologias de ensino e a correlação entre padrões de codificação e desempenho.

\subsection{Análise de Padrões e Autoria}
Com base no perfil de usuário (\textit{user profile}) construído—que inclui frequência de \textit{commits} e uso de estruturas específicas—é possível identificar discrepâncias na autoria de códigos, auxiliando na detecção de plágio e fornecendo uma base para avaliação personalizada.

\section{A Extensão Logado: Proposta de Implementação}
\label{sec:extensao-logado}

A motivação para o desenvolvimento da extensão \textit{Logado} reside na evolução dos ambientes de programação. Argumenta-se que é mais eficaz integrar a funcionalidade de \textit{logging} em uma IDE já consagrada e amplamente utilizada, evitando a curva de aprendizado associada a novas ferramentas. Dessa forma, a extensão será executada diretamente na IDE do estudante, capturando de forma integrada os eventos de desenvolvimento detalhados na seção \ref{sec:metodologia-analise}.