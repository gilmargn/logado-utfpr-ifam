%%%% CAPÍTULO 1 - INTRODUÇÃO
%%
%% Deve apresentar uma visão global da pesquisa, incluindo: breve histórico, importância e justificativa da escolha do tema,
%% delimitações do assunto, formulação de hipóteses e objetivos da pesquisa e estrutura do trabalho.

%% Título e rótulo de capítulo (rótulos não devem conter caracteres especiais, acentuados ou cedilha)
\chapter{Introdução}\label{cap:introducao}


Enquanto o ensino tradicional foca na leitura e escrita de textos, os cursos de programação introduzem uma nova
de forma de pensamento: os algoritmos. Estes, que são sequências lógicas interpretadas por humanos 
e máquinas. Essa disciplina, presente em cursos de áreas tecnológicas, é ``considerada desafiadora para alunos 
e professores" \cite{raabe2005ambiente}, pois exige o desenvolvimento de uma lógica formal.

Ainda segundo \cite{raabe2005ambiente}, as dificuldades no aprendizado de programação podem ser categorizadas
em três grupos distintos, que vão além da simples aptidão individual: Problemas de natureza didática 
(relacionados aos métodos de ensino), cognitiva(relacionados aos desafios de raciocínio lógico e  abstração) e 
afetiva(relacionados à motivação, ansiedade e confiança do aluno). Essa categorização sistêmica ajuda a
entender que o desafio requer iniciativas diferentes.  

No contexto educacional, a avaliação é entendida como uma etapa ao processo de ensino e aprendizagem, o que a 
torna um tema permanente de reflexão e estudo entre especialistas da área, que buscam constantemente refinar 
suas práticas para alcançar melhores resultados \cite{lima2022avaliaccao}

A avaliação no curso técnico é um processo bifronte, contemplando tanto a avaliação da aprendizagem do 
estudante quanto a avaliação do sistema educacional como um todo. No que tange à aprendizagem, o projeto 
pedagógico estabelece que seu objetivo central é a progressão do discente para o alcance do perfil profissional
almejado. Nesse sentido, a Resolução que rege o curso determina que esse processo ``é contínua e cumulativa,
com prevalência dos aspectos qualitativos sobre os quantitativos, bem como dos resultados ao longo do processo 
sobre os de eventuais provas finais" \cite{IFAM2020}. Paralelamente, a avaliação do sistema, referente ao 
rendimento acadêmico, deve ser realizada para cada disciplina individualmente, considerando de forma simultânea 
e obrigatória a frequência do aluno e seu aproveitamento na aquisição de conhecimentos \cite{IFAM2020}.

O ensino de programação é uma atividade multidisciplinar que demanda uma práxis pedagógica apurada -- ou seja, 
uma constante reflexão que integre teoria e prática em sala de aula. Uma estratégia comum é utilizar exemplos 
do cotidiano para estimular a resolução de problemas por meio de algoritmos. Contudo, avaliar a qualidade  
desses algoritmos exige ir além da verificação funcional. 

Esses algoritmos, definidos como procedimentos computacionais que transformam entradas em saídas, geram 
registros conhecidos como logs. Mas como esses logs podem ser utilizados para transformar o ensino e a 
aprendizagem? Este texto explora a interseção entre tecnologia e educação, mostrando como a análise de logs 
pode oferecer perspectivas valiosas para personalizar o ensino e identificar dificuldades dos alunos.

Essas interações, além de transformar entradas em saídas, geram logs. De acordo com Sah (2002, apud 
\cite{clemente2008arquitetura}), ``log é definido como um conjunto de registros com marcação temporal, que 
suporta apenas inserção, e que representa eventos que aconteceram em um computador ou equipamento de rede". 
Além disso, segundo \cite{gu2022logging}, ``um log geralmente contém um grande número de entradas, também 
conhecidas como mensagens de registro". Em geral, o nível do log, seu conteúdo e o local adequado para declará-
lo são decisões tomadas pelos desenvolvedores durante a criação do software.

O processo de ensino pode ser compreendido analogamente a uma tecnologia que pode ser atualizada, adaptada ou 
aplicada em diferentes contextos. No entanto, para estimar o tempo de produção de um código, métodos 
tradicionais de marcação de tempo são insuficientes, sendo necessárias métricas adicionais. Nesse sentido, o 
armazenamento de logs permite coletar informações detalhadas sobre eventos em dispositivos. Como destacado por 
\cite{nguyen_analyzing_2023}, ``esse recurso de previsão possibilita que instrutores identifiquem alunos com 
dificuldades desde o início e ofereçam intervenções direcionadas para apoiar sua jornada de aprendizado''. 
Ademais, a análise de logs pode revelar padrões de comportamento e envio, fornecendo, assim,  dados valiosos 
não apenas para a intervenção individual, mas para a melhoria contínua  e baseada em evidências do processo 
educacional como um todo.

É aqui que entram métricas de análise de código. para além das tradicionais métricas de complexidade 
computacional (como tempo de execução), o professor pode lançar mão de métricas de similaridade e 
adequação estrutural, que avaliam, por exemplo, o uso esperado de variáveis, a aplicação correta de laços de 
repetição e o emprego adequado das palavras reservadas da linguagem. 

O potencial inovador dessa abordagem reside na possibilidade de sistematizar a análise: os dados gerados por 
essas métricas -- coletados em larga escala dos códigos dos estudantes -- podem ser agregados em um dataset. 
Este, por sua vez, permite comparações objetivas não apenas entre indivíduos, mas entre turmas, períodos 
letivos e diferentes metodologias de ensino para a avaliação em larga escala. 


\section{Contexto, motivação e justificativa}\label{sec:consideracoesIniciais}

Diferentemente da avaliação quantitativa, que prioriza resultados 
finais, a avaliação qualitativa é fundamental para aferir o processo de 
aprendizado. Em cursos de algoritmos, esse acompanhamento processual 
pode ser significativamente ampliado com o uso de telemetria.  Conforme 
proposto por \cite{russo2015towards}, essa abordagem é sinérgica em 
relação aos mecanismos tradicionais de \textit{feedback}, como notas em 
laboratórios, tarefas e exames. A telemetria não os substitui, mas 
oferece uma camada adicional de dados contínuos e objetivos sobre o 
comportamento do estudante, enriquecendo a análise do docente.
Saber o que acontece dentro de um sistema enquanto ele está em execução 
é benéfico para fins de depuração, manutenção e para fins de 
análise.\cite{Gatev2021} (tradução nossa). Nosso trabalho buscará 
entender como os discentes produzem os códigos focando em algumas 
observações geradas por logs.

A telemetria proveniente de seus aplicativos geralmente pode ser dividida em três categorias: logs, métricas e rastros. Idealmente, você deve trabalhar para ter um único painel de vidro que possa mostrar todas as três categorias para que você possa navegar entre elas. \cite{Gatev2021}

Para \cite{theLinuxFoundation-2024}``Telemetria é o processo de coleta 
e transmissão automática de dados de sistemas remotos ou distribuídos 
para monitorar, medir e rastrear o desempenho ou o status desses 
sistemas. Os dados de telemetria fornecem percepções em tempo real 
sobre o desempenho de diferentes partes de um aplicativo". 

%KENNEDY, Anna Russo. Towards a Data-Driven Analysis of Programming Tutorials. Telemetry to Improve the Educational Experience in Introductory Programming Courses?, Diss. University of Victoria, 2015.
Apesar do uso consolidado em ferramentas de desenvolvimento de 
software, a aplicação de telemetria no ensino de programação ainda não 
é um prática amplamente utilizada. Seu potencial, no entanto, é 
significativo: as instituições podem empregar os dados telemetrados 
para identificar dificuldades específicas dos alunos na compreensão do 
material, avaliar os nível de desafio apresentado por seus cursos e, 
consequentemente, intervir de forma mais precisa e personalizada.

Diante deste potencial, esta pesquisa tem o intuito de entender o 
processo de construção de algoritmos, as ferramentas e técnicas de 
engenharia de software utilizados no ensino, por meio de logs criados e 
armazenados no processo de ensino-aprendizagem em um IF da região 
Amazônica. Com esta coleta  teremos informações qualiquantitativas 
sobre a produção de códigos, as quais serão fundamentais para a 
formulação de modelos com possibilidades de aplicação no ensino de 
programação e desenvolvimento de soluções para problemas locais. O 
estabelecimento de tal estrutura, permitiria o compartilhamento de 
conhecimentos de maneira cumulativa, podendo ser utilizado e aprimorado 
continuamente por  sucessivas turmas de alunos, de diferentes cursos.

Nesse contexto, conforme destacado por \cite{qian2017students}, 
``esforços para aprimorar o ensino em ciência da computação estão em 
andamento, e os docentes enfrentam desafios significativos em 
disciplinas introdutórias de programação, visando facilitar a 
aprendizagem dos estudantes". 

Nos cursos técnicos em informática e ou afins ofertados pela Rede de 
EPCT, disciplinas como Introdução a Algoritmos, Programação Básica, 
Estrutura de Dados e Programação Orientada a Objetos são frequentemente associadas a altas taxas de dificuldade e evasão. Além dos obstáculos 
cognitivos, os estudantes frequentemente lidam com questões emocionais, 
sentindo-se incapazes de dominar os conceitos e, consequentemente, 
desenvolvendo a crença de que são inaptos para a área de informática 
como um todo. A aplicação da métrica proposta neste estudo permitiria 
uma análise qualiquantitativa do desempenho algorítmico e do uso do 
computador, oferecendo subsídios valiosos para reduzir esses problemas.

Por fim, esta pesquisa reforça o papel central dos IFs, que é promover 
o desenvolvimento socioeconômico local e regional, por meio de 
pesquisas aplicadas e da criação de soluções técnicas e tecnológicas 
alinhadas às demandas da comunidade, fortalecendo os arranjos 
produtivos locais \cite{rodriguesuniversidades}, pois demonstra como o 
corpo docente e técnico buscam alternativas para o aprimoramento do 
ensino e resultados para a instituição. 

Sendo assim, pode-se interpretar esta característica dos IFs como um 
potencializador do uso de plataformas de códigos, estimulando os alunos 
a se dedicarem ao desenvolvimento de soluções para problemas locais, ao 
mesmo tempo permitindo a interação com outras experiências e aportes 
colaborativos diversos.

A aplicação de princípios de Engenharia de Software aliada à análise de 
métricas de uso de ambientes de desenvolvimento integrado (IDE) 
possibilitará avaliações mais robustas. A opção por uma IDE amplamente 
adotada no setor profissional busca uniformizar as ferramentas 
utilizadas no contexto acadêmico, eliminando a dissonância entre 
disciplinas. A solução proposta consiste em uma extensão de operação 
transparente, executando-se em segundo plano sem comprometer a 
experiência de usuários finais (discentes e docentes) durante suas 
atividades de desenvolvimento.

\section{Questões de pesquisa}\label{sec:questoes-de-pesquisas}
\begin{enumerate}
    \item QP1: Quais são os principais objetivos e aplicações do uso de logs no ensino de programação?
    \item QP2: Quais ferramentas, métodos e métricas são mais utilizados para capturar, armazenar e analisar logs em ambientes de ensino de programação?
\end{enumerate}
\section{Objetivos} \label{sec:objetivos}

O objetivo deste trabalho é desenvolver e implementar uma extensão para coleta e análise de logs 
durante o processo de ensino-aprendizagem de programação, com o intuito de fornecer métricas que 
auxiliem na identificação de dificuldades dos alunos, na personalização do ensino e no 
aprimoramento das práticas pedagógicas

\subsection{Objetivos Gerais} \label{subsec:objetivos-gerais}
\begin{enumerate}
	\item O que pretende: Investigar o processo de aprendizagem de 
    programação por meio da análise de logs acadêmicos, identificando 
    padrões e dificuldades comuns entre os estudantes.
	\item Como pretende: Utilizando ferramentas de captura de 
    eventos durante as aulas de programação, processando os dados 
    coletados e organizando-os em um repositório acessível.
	\item Por que pretende: Para melhorar o ensino de programação, 
    oferecendo outras características, objetos de estudos baseados em 
    dados quantitativos e criando um recurso (dataset) que possa ser 
    utilizado por outros pesquisadores e educadores.
\end{enumerate}
\subsection{Objetivos Específicos}\label{subsec:objetivos-especificos}
\begin{enumerate}
	\item Desenvolver uma extensão para registro de eventos durante 
    as aulas de programação.
	\item Implementar o plugin na IDE utilizada na disciplina, com 
    a anuência da instituição e dos estudantes.
	\item Coletar, sincronizar e armazenar dados de eventos gerados 
    durante o experimento.
	\item Desenvolver um dataset com os dados coletados, 
    organizados por turma, instituição e disciplina.
\end{enumerate}

\section{Contribuições esperadas}\label{sec:contribuicoes}

Este trabalho almeja contribuir com os pontos mencionados na seção 1.1 atendendo: 

\begin{enumerate}
    \item Um Mapeamento Sistemático da Literatura sobre uso de logs no ensino de programação focado 
    principalmente nas instituições de ensino brasileiras. 
     \item Desenvolvimento de uma extensão para uma IDE que possa auxiliar na coleta de logs durante o uso do 
     programa e desenvolvimento de códigos.
     \item Desenvolvimento e disponibilidade de um dataset dos logs gerados de forma aberta, respeitando a 
     privacidade e dados sigilosos de todos os participantes. 
\end{enumerate}
\section{Estrutura do trabalho}\label{sec:estrutura-trabalho}

Neste capítulo foram descritos o contexto, motivação e justificativa do trabalhos, suas questões de pesquisa, 
objetivos (geral e específicos), contribuições esperadas e estrutura. No Capítulo 2 é exposta sua fundamentação 
teórica, na qual são definidos os termos e conceitos fundamentais ao entendimento da proposta, bem como 
discorrido acerca dos trabalhos correlatos. Por sua vez, no Capítulo 3 é apresentada a metodologia de pesquisa 
a ser seguida(com suas fases e cronograma) e por fim, são apresentadas os seus apêndices. 