\documentclass[12pt,a4paper]{article}
\usepackage[utf8]{inputenc}
\usepackage[T1]{fontenc}
\usepackage{amsmath}
\usepackage{amssymb}
\usepackage{graphicx}
\usepackage{pgfgantt}


\begin{document}
	\section{Atividade 1: Identificação do problema e motivação}
	
	O projeto é oriundo de pesquisas e orientações relacionadas ao ensino de engenharia de software e programação, áreas que têm ganho destaque no contexto acadêmico e profissional. A pesquisa está vinculada ao mestrado em Computação Aplicada da UTFPR, cuja subárea de Engenharia de Software aborda temas como educação em computação, desenvolvimento de ferramentas educacionais, redução de custos computacionais e financeiros, e melhoria da usabilidade de softwares. A motivação para este estudo surgiu da necessidade de explorar lacunas no ensino de programação, especialmente no que diz respeito à análise de logs acadêmicos e ao uso de ferramentas de captura de eventos durante o processo de aprendizagem.
	
	\section{Atividade 2: Definir os objetivos para uma solução} 
	
	A pesquisa tem grande relevância acadêmica e prática, pois busca preencher uma lacuna ainda pouco explorada no ensino de programação: a análise de logs acadêmicos. Através da captura e análise de eventos durante o processo de desenvolvimento de códigos, será possível:
\begin{itemize}
	\item  Identificar as principais dificuldades enfrentadas pelos estudantes.
	\item Propor melhorias no ensino de programação com base em dados quantitativos.
	\item Criar um dataset de logs acadêmicos que poderá ser compartilhado e utilizado em futuras pesquisas.
	\item Desenvolver ferramentas educacionais mais eficazes, com foco na usabilidade e na redução de custos computacionais.
	
\end{itemize}
	Além disso, o estudo contribuirá para a formação de profissionais mais qualificados na área de informática, preparando-os para os desafios do mercado de trabalho.
	
	\subsection{Objetivos}
	 
	O objetivo deste trabalho é desenvolver e implementar uma extensão para coleta e análise de logs durante o processo de ensino-aprendizagem de programação, com o intuito de fornecer métricas que auxiliem na identificação de dificuldades dos alunos, na personalização do ensino e no aprimoramento das práticas pedagógicas
	
	\subsubsection{Objetivos Gerais}
	\begin{enumerate}
		\item O que pretende: Investigar o processo de aprendizagem de programação por meio da análise de logs acadêmicos, identificando padrões e dificuldades comuns entre os estudantes.
		\item Como pretende: Utilizando ferramentas de captura de eventos durante as aulas de programação, processando os dados coletados e organizando-os em um repositório acessível.
		\item Por que pretende: Para melhorar o ensino de programação, oferecendo outras características, objetos de estudos baseados em dados quantitativos e criando um recurso (dataset) que possa ser utilizado por outros pesquisadores e educadores.
	\end{enumerate}
	\subsubsection{Objetivos Específicos}
	\begin{enumerate}
	\item Desenvolver uma extensão para registro de eventos durante as aulas de programação.
	\item Implementar o plugin na IDE utilizada na disciplina, com a anuência da instituição e dos estudantes.
	\item Coletar, sincronizar e armazenar dados de eventos gerados durante o experimento.
	\item Desenvolver um dataset com os dados coletados, organizados por turma, instituição e disciplina.
\end{enumerate}
	
	\section{Atividade 3: Design e desenvolvimento}
	
	Este trabalho adota uma abordagem metodológica composta por duas etapas principais: uma revisão sistemática da literatura e um survey.
	
	\subsection{Revisão Sistemática da Literatura}
	A revisão sistemática foi conduzida para investigar como os logs têm sido aplicados no ensino de programação. Foram consultadas cinco bibliotecas virtuais (Scopus, IEEE Xplore, SOL SBC, ACM e WILEY), utilizando combinações de palavras-chave e operadores lógicos. A questão central que guiou a pesquisa foi: "Como os logs têm sido utilizados no ensino de programação?". Além disso, foram definidas sub-questões para explorar aspectos específicos, como objetivos, ferramentas, métodos e métricas utilizados. Essa etapa visa mapear o estado da arte e identificar lacunas e oportunidades para o desenvolvimento de novas soluções.
	
	\subsection{Survey}
	A segunda etapa consiste na aplicação de um survey, cujo formulário contém perguntas relacionadas ao uso de ferramentas de log no apoio ao ensino de programação. As questões abordam temas como:
	
\begin{itemize}
	\item O uso de monitoramento no processo de ensino-aprendizagem;
	\item Como os dados de log podem auxiliar na tomada de decisões pedagógicas;
	\item O editor de texto ou IDE mais utilizado (informação essencial para o desenvolvimento de um plugin);
	\item A disposição das instituições em adotar a ferramenta proposta, uma vez disponível.
\end{itemize}
	O ensino de programação não se limita à transmissão de conhecimentos técnicos, mas abrange também aspectos como licenciamento de código, boas práticas de desenvolvimento e integridade acadêmica. Nesse contexto, o combate ao plágio é uma preocupação central, e os logs emergem como um recurso valioso para apoiar a avaliação discente.

	Por meio da análise de padrões de estilo de programação — registrados em um user profile (perfil de usuário) —, é possível identificar discrepâncias na autoria de códigos. Se um trabalho submetido apresenta características incompatíveis com o histórico do estudante (como estrutura, nomenclatura ou lógica), isso pode indicar plágio ou colaboração não autorizada. Além disso, esses dados permitem:
	
	Avaliação personalizada – Acompanhar a evolução individual do discente, detectando dificuldades e adaptando o ensino conforme necessário.
	Geração de datasets para pesquisa – Os logs podem ser organizados em tabelas para estudos empíricos sobre metodologias de ensino, estilos de codificação ou eficácia de ferramentas educacionais.
	Dessa forma, os logs não só reforçam a originalidade do trabalho, mas também enriquecem o processo pedagógico, fornecendo informações em dados.
	
	A análise dos dados coletados pelo plugin será realizada em três etapas principais:  
\begin{enumerate}	
	\item  Coleta e Pré-processamento
	\begin{enumerate}
	\item  Registro de Logs:  
	Os dados brutos são gerados a partir da interação dos estudantes com o ambiente de programação, capturando:  
		\item  Timestamp (data/hora de cada ação);  
			\begin{itemize}
				\item Trechos de código editados;  
				\item  Padrões de estilo (ex.: nomenclatura de variáveis, estrutura de funções);  
				\item   Eventos como compilações, erros e correções.  
			\end{itemize}
		\item Filtragem e Anonimização:  
		\item Dados sensíveis (como nomes ou matrículas discentes) são substituídos por identificadores únicos, garantindo a privacidade.  
	\end{enumerate}
	\item  Análise de Autoria e Detecção de Plágio
		\begin{enumerate}
			\item  Perfil do estudante (*User Profile*):  
			\item  Frequência de commits;  
			\item  Uso de estruturas específicas (ex.: loops, funções).  
		\end{enumerate}
	\item  Geração de Dados Educacionais a partir de eventos
	\begin{enumerate}
		\item  Dataset para Pesquisa
		\item  Eficácia de metodologias de ensino;  
		\item  Correlação entre padrões de codificação e desempenho acadêmico.  
	\end{enumerate}
\end{enumerate}

	\section{Atividade 4: Demonstração}
	
	A pesquisa será realizada no IFAM Campus Boca do Acre, onde um dos pesquisadores atua como professor EBTT na área de informática. O campus disponibilizará a infraestrutura física para a coleta de dados e a realização de experimentos, além de contar com um público-alvo diretamente envolvido com o tema da pesquisa.
	
	O público alvo da pesquisa serão os estudantes matriculados no curso técnico em informática do IFAM Campus Boca do Acre. A amostra será aproximadamente 80 estudantes de dois períodos do curso técnico subsequente em Informática nas disciplinas: Lógica de Programação e algoritmos e Estrutura de Dados. Eventualmente, a pesquisa poderá ser estendida para a comunidade local por meio de cursos de programação ofertados pela instituição, ampliando o escopo de coleta de dados e análise.
	
	A duração serão três semanas(aulas práticas) na infraestrutura do labin (laboratório de informática). A ferramenta utilizada será o Visual Studio.
	
	Todos as etapas seguirão as normas exigidas pelo TCLE e pela Plataforma Brasil. 
\section{Cronograma}
\begin{figure}[h]
	\centering
	\begin{ganttchart}[
		vgrid={*{8}{gray!30, dotted}},
		hgrid,
		x unit=1.5cm,
		y unit title=0.7cm,
		y unit chart=0.6cm,
		title/.style={draw=none, fill=none},
		title label font=\bfseries\footnotesize,
		bar/.style={draw=none, fill=blue!40, rounded corners=2pt},
		bar height=0.5,
		bar label font=\scriptsize,
		bar label node/.append style={left=3pt, text width=3cm, align=right},
		milestone/.style={fill=red!70, shape=rectangle},
		milestone label font=\scriptsize\bfseries
		]{1}{8}
		
		% Year and month titles
		\gantttitle{2025}{5} \gantttitle{2026}{3} \\
		\gantttitlelist{8,9,10,11,12}{1} \gantttitlelist{1,2,3}{1} \\
		
		% Activities
		\ganttbar{Qualificação}{1}{1} \\
		\ganttbar{Experimento 1}{2}{3} \\
		\ganttbar{Experimento 2}{3}{4} \\
		\ganttbar{Conclusão da pesquisa}{5}{5}\\
		\ganttbar{Submissão de artigos}{6}{7}\\
		\ganttbar{Escrita da dissertação}{1}{8} \\
		\ganttmilestone{Entrega da dissertação}{7}{8}
		
	\end{ganttchart}
	\caption{Cronograma de pesquisa (Agosto 2025 - Março 2026)}
	\label{fig:gantt}
\end{figure}
%
%\begin{figure}[h]
%	\centering
%	\resizebox{\textwidth}{!}{%
%		\begin{ganttchart}[
%			x unit=0.8cm,
%			y unit title=0.8cm,
%			y unit chart=0.7cm,
%			title/.style={draw=none, fill=none},
%			title label font=\bfseries\footnotesize,
%			bar/.style={draw=none, rounded corners=3pt},
%			bar height=0.6,
%			bar label font=\footnotesize,
%			group/.style={draw=black, fill=gray!20},
%			group height=0.5,
%			group peaks height=0.3,
%			milestone/.style={fill=red, shape=diamond},
%			milestone label font=\footnotesize\bfseries,
%			link/.style={-latex, thick, blue!50!black},
%			link bulge=2
%			]{8}{15} % Months 8 (Aug) to 15 (Mar next year)
%			
%			\gantttitle{Project Timeline (August=8)}{8} \\
%			\gantttitlelist{8,...,15}{1} \\
%			
%			% Phase 1: Qualification and Experiments
%			\ganttgroup{Pesquisa}{8}{11} \\
%			\ganttbar[bar/.append style={fill=phase1}]{Qualificação}{8}{8} \\
%			\ganttbar[bar/.append style={fill=phase1}, name=exp1]{Experimento 1}{9}{10} \\
%			\ganttbar[bar/.append style={fill=phase2}, name=ajust]{Ajustes}{11}{11} \\
%			\ganttbar[bar/.append style={fill=phase1}, name=exp2]{Experimento 2}{11}{11} \\
%			
%			% Phase 2: Conclusion and Writing
%			\ganttgroup{Escrita}{12}{15} \\
%			\ganttbar[bar/.append style={fill=phase3}, name=conc]{Conclusão}{12}{12} \\
%			\ganttbar[bar/.append style={fill=phase3}, name=artigo]{Artigo}{1}{2} \\
%			\ganttbar[bar/.append style={fill=phase3}, name=escrita]{Escrita da Dissertação}{8}{15} \\
%			\ganttmilestone[name=defesa]{Defesa}{3} \\
%			
%			% Phase 3: Evaluation
%			\ganttgroup{Avaliações}{9}{13} \\
%			\ganttbar[bar/.append style={fill=phase2}]{Coleta de Dados}{9}{10} \\
%			\ganttbar[bar/.append style={fill=phase2}]{Análise}{10}{12} \\
%			\ganttbar[bar/.append style={fill=phase2}]{Relatório}{12}{13} \\
%			
%			% Dependencies
%			\ganttlink[link type=f-s]{exp1}{ajust}
%			\ganttlink[link type=f-s]{ajust}{exp2}
%			\ganttlink[link type=f-s]{exp2}{conc}
%			\ganttlink[link type=f-s]{conc}{artigo}
%			\ganttlink[link type=f-s]{artigo}{defesa}
%		\end{ganttchart}
%	}
%	\caption{Cronograma do Projeto (Agosto=8 até Março=15)}
%	\label{fig:gantt}
%\end{figure}
\end{document}