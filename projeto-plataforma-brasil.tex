\documentclass[10pt,a4paper]{article}
\usepackage[utf8]{inputenc}
\usepackage{amsmath}
\usepackage{amsfonts}
\usepackage{amssymb}
\usepackage{graphicx}
\usepackage[left=2cm,right=2cm,top=2cm,bottom=2cm]{geometry}
\author{Gilmar Gomes do Nascimento}
\title{PROJETO DE PESQUISA ENVOLVENDO SERES HUMANOS}
\begin{document}
\begin{center} PROJETO DE PESQUISA ENVOLVENDO SERES HUMANOS \end{center}
\begin{enumerate}
\item \textbf{Apresentação}\\
\textbf{Título da Pesquisa:} Uso de logs no ensino de programação\\
\textbf{Nome do Pesquisador Principal:}Laudelino Cordeiro Bastos\\ % No CEP-UTFPR deve ser o orientador.
\textbf{Nome dos demais Membros da Equipe de Pesquisa:} Laudelino Cordeiro Bastos, Maria Claudia Figueireido Pereira Emer, Adolfo Gustavo Serra Seca Neto, Gilmar Gomes do Nascimento\\
\textbf{Instituição Proponente:} UTFPR\\
\item \textbf{Área de Estudo} \\
Não se aplica. A pesquisa em ensino de programação com análise de logs se enquadra na área de \textbf{Ciências Exatas e da Terra}, especificamente em Computação Aplicada ao Ensino.

%\textbf{Grandes Áreas do Conhecimento (CNPq) (Selecione até três):}
%\begin{itemize}
%\item \textbf{Grande área 1. Ciências exatas e da terra}
%\item Grande área 2. Ciências biológicas
%\item Grande área 3. Engenharias
%\item Grande área 4. Ciências da saúde
%\item Grande área 5. Ciências agrárias
%\item Grande área 6. Ciências sociais aplicadas (Carta circular – 110/2017)
%\item Grande área 7. Ciências humanas
%\item Grande área 8. Linguística, letras e artes
%\item Grande área 9. Outros
%Propósito Principal do Estudo (OMS):
%\item Clínico – (Ao selecionar esta opção serão habilitadas as opções: "Acrônimo do Título Público", "Expansão do Acrônimo do
%Público", “Acrônimo”, "Expansão do Acrônimo", “Múltiplos ID's Secundários” )
%\item Ciências básicas,
%\item Ciências sociais, humanas ou filosofia aplicadas à saúde – normalmente as nossas
%\item Saúde Coletiva / Saúde Pública - normalmente as nossas
%\item Supportive care – cuidados de enfermagem para prevenir, controlar e aliviar condições clínicas do paciente,
%\item Outros.
%\end{itemize}
\textbf{Título Público da Pesquisa:} Uso de log no ensino de programação\\
\textbf{Título Principal da Pesquisa:} Uso de log no ensino de programação \\
\textbf{Contato Público:} Gilmar Gomes do Nascimento\\
\textbf{Contato Científico:} Laudelino Cordeiro Bastos
\item \textbf{Desenho de Estudo / Apoio Financeiro}\\
\noindent \textbf{Desenho}: 
O projeto é proveniente de pesquisas e orientações relacionadas ao ensino de engenharia de software e programação, áreas que têm ganhado destaque no contexto acadêmico e profissional. A pesquisa está vinculada ao mestrado em Computação Aplicada da UTFPR, cuja subárea de Engenharia de Software aborda temas como educação em computação, desenvolvimento de ferramentas educacionais, redução de custos computacionais e financeiros, e melhoria da usabilidade de softwares. A motivação para este estudo surgiu da necessidade de explorar lacunas no ensino de programação, especialmente no que diz respeito à análise de logs acadêmicos e ao uso de ferramentas de captura de eventos durante o processo de aprendizagem.

Este estudo caracteriza-se como uma pesquisa-ação, com abordagem quantitativa, que visa intervir no processo de ensino-aprendizagem para, simultaneamente, investigar o fenômeno e propor melhorias. A unidade de análise são os eventos de interação capturados pelos logs da ferramenta educacional durante sessões de programação.

A população-alvo são estudantes de programação de nível técnico. A amostra será não probabilística, por conveniência, composta por aproximadamente 80 (oitenta) estudantes do curso técnico em Informática do Instituto Federal do Amazonas (IFAM) do campus Boca do Acre. Os critérios de inclusão serão: (1) estar matriculado em uma disciplina de programação e (2) consentir formalmente com a participação por meio do Termo de Consentimento Livre e Esclarecido (TCLE). Dados demográficos como idade e sexo serão coletados via questionário inicial para caracterização da amostra.

A participação terá a duração total de 4 (quatro) semanas, correspondendo a um módulo específico da disciplina. Os participantes serão conduzidos a um laboratório de informática do IFAM, onde:

\begin{enumerate}
\item Assinarão o TCLE e responderão a um questionário de perfil.
\item Instalarão, via aba de "Extensões", a ferramenta de captura de logs em um editor de código predefinido.
\item Realizarão uma série exercícios de programação com complexidade crescente, durante sessões semanais de 2 horas cada.
\item A ferramenta coletará automaticamente dados sobre uso de palavras reservadas da linguagem de programação e o \textit{timestamp}{carimbo do tempo}.
\end{enumerate}

Os logs brutos serão processados e agregados para análise estatística descritiva e inferencial, visando a construção de um conjunto de dados (\textit{dataset}) para posteriormente, identificar padrões de código e correlações entre as métricas de interação e o desempenho na resolução dos exercícios.

%Parte da metodologia que se refere a participação com os seres humanos (Quem vai participar da pesquisa? Idade?
%Sexo? Quantas vezes eles devem participar da pesquisa? Qual o tempo total da participação? O que eles devem fazer? Etc...)
%O conceito de Desenho de estudo envolve a identificação do tipo de abordagem metodológica que se utiliza para responder a
%uma determinada questão, implicando, assim, a definição de certas características básicas do estudo, como: a população e a
%amostra estudadas; a unidade de análise; a existência ou não de intervenção direta sobre a exposição; a existência e tipo de
%seguimento dos indivíduos, entre outras.
%(Para pesquisas da área social pode ser incluído neste campo: vide metodologia segundo a carta circular %110/2017). \\
\textbf{Financiamento:}
Tipo de Financiamento: Financiamento próprio.
%Se existir financiamento:
%CNPJ \\
%Nome \\
%E-mail \\
%Telefone \\
Palavras- Chave: ensino de programação; log; telemetria; logging\\
\item \textbf{Detalhamento do Estudo} \\
\noindent \textbf{Resumo:}
No desenvolvimento de software, é comum utilizar métricas para avaliar a qualidade do código e o tempo despendido em sua produção. No ambiente profissional, essa análise ocorre principalmente por meio de pull requests, enquanto no contexto acadêmico, a avaliação costuma ser feita por meio de envios pontuais (como e-mails ou sistemas de gestão de aprendizagem - LMS). Embora os LMS registrem logs de atividades e eventos, há uma lacuna na coleta estruturada de dados sobre o processo de aprendizagem em programação, especialmente em
ambientes como laboratórios de informática. Este trabalho propõe a criação de um dataset com registros (logs) do processo de desenvolvimento de código, capturando o rastro da aprendizagem em tempo real. Diferentemente de abordagens convencionais, que dependem de Sistemas de Gestão de Aprendizagem, a coleta será realizada por meio de um plugin integrado a um editor de código amplamente utilizado, voltado tanto para programação quanto para documentação. Espera-se fornecer uma base de dados sobre o ensino-aprendizagem de programação que possa auxiliar na identificação de dificuldades dos estudantes, na personalização do ensino e no aprimoramento das práticas pedagógicas.
%Contextualização, problema, objetivo, metodologia, resultados esperados.
\\ \noindent\textbf{Introdução:} 
Enquanto o ensino tradicional prioriza a leitura e a escrita como ferramentas para interpretação e produção de textos, os cursos de programação introduzem uma nova linguagem: os algoritmos. Esses algoritmos, que são sequências lógicas de instruções executáveis por máquinas, representam um tipo de texto único, interpretado tanto por humanos quanto por computadores. Durante o desenvolvimento desses algoritmos, são gerados registros automáticos, conhecidos como logs, que capturam detalhes sobre o processo de criação, execução e depuração do código. Esses registros podem ser transformados em dados valiosos para entender como os estudantes aprendem e interagem com a programação.
Apesar do potencial dos logs para revolucionar o ensino de programação, ainda há desafios significativos na coleta, organização e interpretação desses dados. Muitas vezes, as ferramentas existentes não são adaptadas ao contexto educacional, limitando sua utilidade para professores e alunos. Este trabalho propõe investigar como os logs podem ser integrados de forma mais eficiente ao ensino de programação, explorando métodos inovadores para coletar e analisar esses dados, bem como transformar essas informações em ações práticas e direcionadas.
A relevância dessa abordagem vai além da sala de aula. Ao utilizar logs para monitorar o progresso dos alunos, é possível criar um ciclo de feedback contínuo, onde os instrutores podem identificar padrões de dificuldades, adaptar o conteúdo às necessidades individuais e promover uma aprendizagem mais eficiente. Além disso, a análise de logs pode contribuir para a pesquisa em educação computacional, fornecendo evidências empíricas sobre como os estudantes aprendem programação e quais estratégias pedagógicas são mais eficazes.


%Contextualização, lacuna/problema de pesquisa, o que vai ser feito (quais as contribuições? O que seu trabalho
%apresenta de diferente dos demais), evidências que o que está sendo proposto pode funcionar.
\textbf{Hipótese:}
Os resultados provenientes desta pesquisa consistirão em dados de telemetria que poderão gerar novas investigações, conceitos e até mesmo avanços no entendimento pedagógico do ensino de programação. Além disso, diversas subáreas da computação poderão se beneficiar desses resultados, como ciência de dados, bancos de dados não relacionais e outras áreas interdisciplinares. 

Os frutos desta pesquisa incluirão informações detalhadas sobre o uso de computadores e ferramentas computacionais em sala de aula, contribuindo também para campos como pedagogia e psicologia. Adicionalmente, os estudantes envolvidos serão introduzidos a conceitos de telemetria e ao trabalho monitorado por computadores, ampliando seu conhecimento técnico e prático. A utilização de telemetria no contexto educacional representa uma inovação significativa, abrindo caminho para metodologias de ensino mais personalizadas e eficientes.

Essa abordagem não só aprimora o ensino de programação, mas também prepara os estudantes para um mercado de trabalho cada vez mais orientado por dados e automação.


% É uma possível resposta ao seu objetivo primário. É uma afirmação.
%(Para pesquisas na área social: Carta Circular 110/2017 – se não compreender a elaboração desse item escrever: Não se aplica).
 \noindent \textbf{Objetivo Primário:} (objetivo geral):
O objetivo geral deste trabalho é implementar uma extensão para coleta e análise de logs durante o processo de ensino-aprendizagem de programação, com o intuito de fornecer métricas que auxiliem na identificação de dificuldades dos alunos, na personalização do ensino e no aprimoramento das práticas pedagógicas.
 
 
% O que será feito? É uma frase (com apenas um verbo)
 \noindent \textbf{Objetivos Secundários:} 
\begin{enumerate}
\item Desenvolver uma extensão para registro de eventos durante as aulas de programação.
\item Implementar o plugin na IDE utilizada na disciplina, com a anuência da instituição e dos estudantes.
\item Coletar, sincronizar e armazenar dados de eventos gerados durante o experimento.
\item Desenvolver um dataset com os dados coletados, organizados por turma, instituição e disciplina.

\end{enumerate}

%(objetivos específicos, normalmente de três a cinco, são os passos que você irá percorrer para alcançar o
%objetivo geral).
 \noindent \textbf{Metodologia Proposta:}\\
%Qual o tipo de pesquisa quanto ao objetivo? (Exploratória? Descritiva? Explicativa?)\\
Este trabalho adota uma abordagem metodológica mista, combinando Revisão Sistemática da Literatura  e um survey exploratório com a metodologia de Design Science Research para o desenvolvimento do artefato proposto. 
%Qual o tipo de pesquisa quanto a abordagem? (Quantitativa? Qualitativa?).\\
Para mapear o estado da arte sobre o uso de logs no ensino de programação, foi conduzida uma Revisão Sistemática da Literatura (RSL).Complementarmente à RSL, foi realizado um survey(online) com 14 participantes, docentes, mestrandos em computação e pesquisadores de computação em foco no ensino de programação com o objetivo de investigar o uso de ferramentas de capturas de logs e construção de avaliações quantitativas e desenvolvimento de dataset acadêmicos. O instrumento de coleta continha 10 questões abordando logs, uso de ferramentas e IDEs. O Design Science Research é adequado para pesquisas que visam desenvolver e avaliar artefatos que resolvam problemas identificados em contextos organizacionais e educacionais.


%O que vai ser feito? Descreva na ordem que cada etapa será realizada.\\
%Qual a população?\\
%Qual a amostra? Como ela foi escolhida/calculada (apresentar o cálculo)?\\
%Como será o recrutamento?\\
%Como será a participação dos seres humanos? (Transversal? Longitudinal?)\\
%Como vai ser feito?\\
%Que equipamentos serão utilizados (modelos/marcas)?\\
%Serão utilizados equipamentos descartáveis? Descrever. \\
%Como será higienizado o ambiente? Macas? Equipamentos que as pessoas terão contato?
%Que instrumentos serão utilizados? (Ex: questionário? Anamnese? Entrevista semi estruturada? Entrevista estruturada?) – OBS:
%Lembre que todo o instrumento utilizado deve ser anexado na Plataforma Brasil ao final da submissão do projeto.
%Haverá uso de placebo ou a existência de grupos que não serão submetidos a nenhuma intervenção? Se sim, como e quando esse grupo terá acesso aos benefícios do projeto?
%Haverá aplicação de washout? Se sim, explique como e por quanto tempo.


\textbf{Critério de Inclusão:} %Que critérios você vai utilizar para compor sua amostra?

\textbf{Critério de Exclusão:} Dentre as pessoas acima, que critérios você vai utilizar para identificar pessoas que não irão participar do seu estudo? Ex: pessoas que estiverem de férias? que não comparecerem a 75\% das atividades propostas?
OBS: Não é o oposto da inclusão. Para ser excluída, a pessoa tem que primeiramente ter sido incluída.
OBS: As vezes, quando a inclusão está muito delimitada, eles não existem. Neste caso escrever: Não se aplica.
\textbf{Riscos:} %Descreva os riscos e as formas que você vai utilizar para minimizar os mesmos.
%OBS: Toda pesquisa com seres humanos envolve risco em tipos e gradações variados. Mesmo a aplicação de um questionário pode gerar constrangimento. Quanto maiores e mais evidentes os riscos, maiores devem ser os cuidados para minimizá-los e a proteção oferecida pelo Sistema CEP/CONEP aos participantes

\textbf{Benefícios:} % Descreva os benefícios para as pessoas que irão participar da sua pesquisa. Se não existir um benefício direto, deixe isso claro e descreva os benefícios para ciência e/ou comunidade.
%OBS: São admissíveis pesquisas cujos benefícios a seus participantes forem exclusivamente indiretos, desde que consideradas as dimensões física, psíquica, moral, intelectual, social, cultural ou espiritual desses.

O ensino de programação não se limita à transmissão de conhecimentos técnicos, mas abrange também aspectos como licenciamento de código, boas práticas de desenvolvimento e integridade acadêmica. Nesse contexto, o combate ao plágio é uma preocupação central, e os logs emergem como um recurso valioso para apoiar a avaliação discente.
Por meio da análise de padrões de estilo de programação — registrados em um user profile (perfil de usuário) —, é possível identificar discrepâncias na autoria de códigos. Se um trabalho submetido apresenta características incompatíveis com o histórico do estudante (como estrutura, nomenclatura ou lógica), isso pode indicar plágio ou colaboração não autorizada. Além disso, esses dados permitem: Avaliação personalizada – Acompanhar a evolução individual do discente, detectando dificuldades e adaptando o ensino conforme necessário.
Geração de datasets para pesquisa – Os logs podem ser organizados em tabelas para estudos empíricos sobre metodologias de ensino, estilos de codificação ou eficácia de ferramentas educacionais.
Dessa forma, os logs não só reforçam a originalidade do trabalho, mas também enriquecem o processo pedagógico, fornecendo informações em dados.

Metodologia de Análise de Dados: %Como você vai analisar os dados? Ex: que testes estatísticos serão utilizados? Ou que tipo de análise de conteúdo será realizada?
%OBS: (Para pesquisas na área social: Carta Circular 110/2017 – Se já estiver contemplado no item metodologia proposta informar:
A análise dos dados coletados será realizada em três etapas principais:
\begin{enumerate}
\item Coleta e Pré-processamento
Os dados brutos são gerados a partir da interação dos estudantes com o ambiente de programação, capturando eventos como edições de código, compilações e erros. Esta fase envolve:
		\begin{itemize}
   		\item \textbf{Registro de Logs}: Captura de \textit{timestamp}, trechos de código editados e 						eventos de desenvolvimento.
  		\item \textbf{Filtragem e Anonimização}: Substituição de dados sensíveis por identificadores únicos 				para garantir a privacidade.
	\end{itemize}
\item Geração de Indicadores Educacionais
	Os \textit{logs} brutos são transformados em métricas educacionais significativas (e.g., tempo para 			resolver um erro, padrões de tentativa-e-erro), criando um \textit{dataset} para pesquisa. Este 				\textit{dataset} permitirá estudos sobre a eficácia de metodologias de ensino e a correlação entre 				padrões de codificação e desempenho.
\item Análise de Padrões e Autoria
	Com base no perfil de usuário (\textit{user profile}) construído que inclui frequência de 						\textit{commits} e uso de estruturas específicas—é possível identificar discrepâncias na autoria de 			códigos, auxiliando na detecção de plágio e fornecendo uma base para avaliação personalizada.
\end{enumerate}
não se aplica).
Desfecho Primário: Resultados esperados em relação ao objetivo primário.
OBS: Para pesquisas na área social: Carta Circular 110/2017: escrever: não se aplica.
Desfecho Secundário: Resultados esperados em relação aos objetivos secundários.
OBS: Para pesquisas na área social: Carta Circular 110/2017: escrever: não se aplica.
Tamanho da Amostra no Brasil:
OBS: Para pesquisas na área social - Carta Circular 110/2017: o pesquisador deverá inserir o número 0. Mas na metodologia
proposta devem estar incluídos os critérios para definição dos participantes da pesquisa.
Data do Primeiro Recrutamento: (pode ser a mesma do início da coleta de dados)
\item \textbf{Outras Informações}
Haverá uso de fontes secundárias de dados?
Sim – clique em sim, se você for utilizar prontuários por exemplo .
Não – clique em não, se você for coletar todos os dados de que precisa.
Informe o número de indivíduos que serão abordados pessoalmente, recrutados, ou que sofreram algum tipo de intervenção
neste centro de pesquisa:
Grupos em que serão divididos os sujeitos de pesquisa neste centro: (adicionar grupo)
ID Grupo
No.de Indivíduos – somados deve ser igual
ao do número informado para tamanho da
amostra
Intervenções a serem relizadas
Descreva as intervenções: Ex: Anamnese,
medidas antropométricas, etc...
OBS: Carta Circular 110/2017: se foi informado 0 na amostra, deve-se manter o zero,aqui nos grupos também.
\\ \textbf{O Estudo é Multicêntrico no Brasil?}
Sim ou Não
Se sim quais serão os demais centros co-partipantes?
São as que participam de projetos elaborados por uma Instituição de Pesquisa, mas que são aplicados simultaneamente em outras,
seguindo exatamente o mesmo procedimento experimental, porém com um pesquisador responsável em cada Instituição; ou
aquela na qual haverá o desenvolvimento de alguma etapa da pesquisa (CARTA Nº 0212/CONEP/CNS). – Cuidar com o prazo para o
início da coleta – deve passar por mais de um CEP tem que prever o prazo para isso.
\\ \textbf{Propõe dispensa do TCLE?}
Não
Se sim, inclua a justificativa.
OBS: Carta Circular 110/2017: o pesquisador é obrigado a inserir o documento de garantias que será entregue ao participante
Haverá retenção de amostras para armazenamento em banco?
Sim ou Não
OBS: Carta Circular 110/2017: deve assinalar este campo com não, porque não podem coletar amostras neste tipo de pesquisa.
\\ \textbf{Cronograma de Execução:} \\%(adicionar cronograma)
\end{enumerate}
\end{document}